% ----------------------------------------
%
% serpentTools presentation
% Given at PHYSOR18 - April 22nd
%
% Author: Andrew Johnson
% This document and the PDF produced
% are licensed under the MIT License
% found in LICENSE in the parent directory 
%
% ----------------------------------------

\setbeamertemplate{bibliography item}{\insertbiblabel}

\title{The serpent-tools python package}
\subtitle{A collection of data parsers for interacting with SERPENT outputs}
\author{Andrew Johnson}
\institute[PHYSOR 2018]{Georgia Institute of Technology}
\date{22 April, 2018}

\newcommand{\sss}{\texttt{SERPENT }}
\newcommand{\colShare}{0.48\textwidth}
% add slides to the appendix without increasing total slide count
% https://tex.stackexchange.com/a/2559

\newcommand{\backupbegin}{
       \newcounter{framenumberappendix}
          \setcounter{framenumberappendix}{\value{framenumber}}
      }
\newcommand{\backupend}{
     \addtocounter{framenumberappendix}{-\value{framenumber}}
        \addtocounter{framenumber}{\value{framenumberappendix}} 
    }  

% link options
\definecolor{links}{HTML}{2A1B81}
\hypersetup{colorlinks,linkcolor=,urlcolor=links}
\newcommand{\toapi}[3]{\href{https://serpent-tools.readthedocs.io/en/latest/api/#1.html\##2.#3}{\texttt{#3}}}
\newcommand{\github}[1]{\url{https://github.com/CORE-GATECH-GROUP/serpent-tools/#1}}
\newcommand{\ghissue}[1]{\href{https://github.com/CORE-GATECH-GROUP/serpent-tools/issues/#1}{GH\##1}}
\newcommand{\ghpull}[1]{\href{https://github.com/CORE-GATECH-GROUP/serpent-tools/pulls/#1}{GH\##1}}
\begin{document}

\begin{frame}
\titlepage
\end{frame}

\mode<beamer>{
\begin{frame}{Outline}
\tableofcontents
\end{frame}
}
\section{Overview and Scope}

\mode<beamer>{\subsection{Motivation}
\begin{frame}{Motivation}
    Variety of outputs produced
    \begin{itemize}
        \item Plain text outputs - many MATLAB-compatible 
        \item Some are produced per time-step
    \end{itemize}
    Benefits of MATLAB-syntax\footnote{Similarly for freeware such as Octave}
    \begin{itemize}
        \item Files loaded into workspace automatically
        \item Little/no conversion for MATLAB scripts
        \item \textbf{Resource consumption for large files}
    \end{itemize}
\end{frame}

\begin{frame}{Motivation}
    \begin{columns}[T]
        \begin{column}{\colShare}
            Design an alternative set of parsing utilities
            \begin{itemize}
                \item Maintain all data stored in outputs
                \item Present data in a logical framework
            \end{itemize}
        \end{column}
        \onslide<2->
        \begin{column}{\colShare}
            Our requirements
            \begin{itemize}
                \item Easy learning curve
                \item Flexible operation
                    \begin{itemize}
                        \item Store a subset of all data
                    \end{itemize}
                \item Simplify and automate routine analyses 
            \end{itemize}
        \end{column}
    \end{columns}
\end{frame}
}

\subsection{serpentTools}
\begin{frame}{serpentTools}
    \begin{columns}[T]
        \begin{column}{\colShare}
            \begin{itemize}
                \item Collection of parsing tools and data structures
                \item Object-oriented Python
                    \mode<beamer>{
                        \begin{itemize}
                            \item Supports 2.7 and 3.5+
                        \end{itemize}
                    }
                \item High-level of control over processing
                \item Utilizes standard mathematical packages
                    \mode<beamer>{
                        \begin{itemize}
                            \item numpy\footnotemark[1], matplotlib\footnotemark[2]
                        \end{itemize}
                    }
            \end{itemize}
        \end{column}
        \mode<beamer>{
            \footnotetext[1]{\cite{numpy-scipy}}
            \footnotetext[2]{\cite{Hunter:2007}}
            \begin{column}{\colShare}
                \begin{block}{}
                Aim to make interacting with \sss outputs flawless and easily scriptable
            \end{block}
                \begin{figure}
                    \centering
                    \includegraphics[width=\textwidth]{./figures/cover.png}
                \end{figure}
            \end{column}
        }\mode<handout>{
            \begin{column}{\colShare}
                \begin{itemize}
                    \item Quick and lightweight
                    \item Python is an excellent scripting language
                    \item Open source
                \end{itemize}
            \end{column}
        }
    \end{columns}
\end{frame}

\mode<beamer>{
    \begin{frame}{Why serpentTools?}
        \begin{itemize}
            \item Quick and lightweight
            \item Full control over what data is obtained from file
            \item Python is great for scripting and passing data between programs
            \item Wide support for mathematics and data analysis with external libraries
        \item Compatibility between \sss versions - \ghissue{117}
            \item Open source development
        \end{itemize}
    \end{frame}
}

\subsection{Resources}

\begin{frame}{Resources}
    \begin{itemize}
        \item Available on GitHub at \github{}
        \item \href{https://github.com/CORE-GATECH-GROUP/serpent-tools/blob/master/LICENSE}{Permissive MIT license}
        \item Brief installation guide
            \begin{itemize}
                \item Download the latest release as a zipped file: \url{https://github.com/CORE-GATECH-GROUP/serpent-tools/releases/latest}
                \item Unzip/extract the files
                \item Run \\ \texttt{python setup.py install --user}
            \end{itemize}
        \item Documentation available at \url{http://serpent-tools.readthedocs.io/en/latest}
        \item<handout> Examples in manual: \url{http://serpent-tools.readthedocs.io/en/latest/examples/index.html}
        \mode<handout>{\item Demonstration outputs and notebook can be found at \url{https://github.com/CORE-GATECH-GROUP/physor18-serpentToolsDemo} 
}
    \end{itemize}
\end{frame}

\section{Demo and Walkthrough}
\mode<beamer>{
    % demo slides for beamer presentation
\begin{frame}{Jupyter notebooks}
    \begin{itemize}
        \item Demonstration outputs and notebook can be found at \url{https://github.com/CORE-GATECH-GROUP/physor18-serpentToolsDemo} 

        \item Examples in manual: \url{http://serpent-tools.readthedocs.io/en/latest/examples/index.html}
        \item Utilizes jupyter notebooks\footnotemark[1]
            \begin{itemize}
                \item Interactive python sessions
                \item More at \url{https://jupyter.org/install}
            \end{itemize}
    \end{itemize}
    \footnotetext[1]{\cite{Kluyver:2016aa}}
\end{frame}

\subsection{Detector files}
\begin{frame}{Detector file}
    \begin{itemize}
        \item \toapi{detector}{serpentTools.parsers.detector}{DetectorReader} objects used to parse detector files
        \item Support detectors with complex bin structures
            \begin{itemize}
                \item energy groups, reactions, Cartesian mesh, etc
            \end{itemize}
        \item Detector data is stored in \toapi{containers}{serpentTools.objects.containers}{Detector} objects
        \item Simple spectral and mesh plot routines
        \item Read multiple files using \toapi{samplers}{serpentTools.samplers.detector}{DetectorSampler}
    \end{itemize}
\end{frame}

\subsection{Depletion files}
\begin{frame}{Depleted material files}
    \begin{itemize}
        \item Parse through the file with a \toapi{depletion}{serpentTools.parsers.depletion}{DepletionReader}
        \item Store metadata such as isotopes, and burnup schedule
        \item Option to store select materials and/or select variables
            \begin{itemize}
                \item store atomic density only for materials containing phrase \texttt{fuel} 
            \end{itemize}
        \item Plot quantities against burnup or days, for some or all isotopes
        \item Read multiple files using \toapi{samplers}{serpentTools.samplers.depletion}{DepletionSampler} to obtain average quantities and uncertainties
    \end{itemize}
\end{frame}

\subsection{Other Functionality}
\begin{frame}{Other Functionality}
    \begin{itemize}
        \item \toapi{branching}{serpentTools.parsers.branching}{BranchingReader} capable of parsing output file
        \item Bare-bones \toapi{history}{serpentTools.parsers.history}{HistoryReader} scrapes arrays from file
        \item Read \toapi{parsers}{serpentTools.parsers}{depmtx} files containing isotopics and depletion matrices
    \end{itemize}
\end{frame}

}
\section{Closing remarks}
\mode<beamer>{
\begin{frame}{Status}
    \begin{columns}[T]
        \begin{column}{\colShare}
            In-progress features
        \begin{itemize}
            \item XSPlot file reader - \ghpull{95}
            \item Results file reader - \ghpull{91}
            \item Sensitivity file reader - \ghissue{13}
        \end{itemize}
        \end{column}
        \begin{column}{\colShare}
            Coming down the pipe
            \begin{itemize}
                \item Fission matrix file reader - \ghissue{5}
                \item Increased support for \sss versions - \ghissue{117}
            \end{itemize}
        \end{column}
    \end{columns}
\end{frame}

\subsection{Conclusion}
\begin{frame}{Conclusion}
    \begin{itemize}
        \item Introduced the \texttt{serpentTools} python package for interacting with \sss outputs
        \item Demonstrations of currently supported file types
            \begin{itemize}
                \item Access to all of the same data or a subset
                \item Simple visualization routines
            \end{itemize}
    \end{itemize}
\end{frame}
}
\subsection{Acknowledgments}
\begin{frame}{Acknowledgments}
    \begin{columns}[T]
        \begin{column}{\colShare}
            Contributors
            \begin{itemize}
                    \item Dr. Dan Kotlyar
                    \item Stefano Terlizzi
                    \item Gavin Ridley\footnotemark[1]
            \end{itemize}
        \end{column}
        \begin{column}{\colShare}
            Financial support for this work came from the NRC 
        \end{column}
    \end{columns}
    \footnotetext[1]{Univ. Tennessee Knoxville}
\end{frame}

\mode<beamer>{
    % outro slides for beamer presentation
\begin{frame}{References}
    \bibliography{refs.bib}
    \bibliographystyle{apalike}  
\end{frame}

\begin{frame}{Thank you!}
    Questions?
\end{frame}

\appendix
\backupbegin

\subsection{Contributing}
\begin{frame}{Contributing}
    \begin{itemize}
        \item Contributions are welcome!
            \begin{itemize}
                \item \href{https://serpent-tools.readthedocs.io/en/latest/develop/index.html}{Developer guide in docs}
            \end{itemize}
        \item Eager to expand the functionality to support other file types and \sss versions
        \item Issues can be added to the \href{https://github.com/CORE-GATECH-GROUP/serpent-tools/issues}{Issue board}
    \end{itemize}
\end{frame}

\subsection{User Control}
\begin{frame}{User Control}
    \begin{itemize}
        \item Many readers support settings that control what data is stored
        \item Select individual and/or group of results data
        \item Full settings listed in documentation
            \begin{itemize}
                \item \href{http://serpent-tools.readthedocs.io/en/latest/settingsTop.html\#default-settings}{All default settings}
                \item \href{http://serpent-tools.readthedocs.io/en/latest/variableGroupsTop.html}{Variable groups}
            \end{itemize}
        \item Examples in documentation
            \begin{itemize}
                \item \href{http://serpent-tools.readthedocs.io/en/latest/examples/Branching.html\#branching-settings}{Settings for BranchingReader}
                \item \href{http://serpent-tools.readthedocs.io/en/latest/examples/DepletionReader.html\#depletion-settings}{Settings for DepletionReader}
            \end{itemize}
    \end{itemize}
\end{frame}

\subsection{Branching files}
\begin{frame}{Branching file}
    \begin{itemize}
        \item Recent \sss versions create coefficient files from an automated burnup sequence 
            \begin{itemize}
                \item Vary physical properties such as materials, temperature, and soluble boron
                \item Reaction cross sections and other group constants
            \end{itemize}
            \note[item]{21 MB file with 81 branch points - 9 BU and 9 variations}
            \note[item]{Read in <10 micro seconds}
        \begin{itemize}
            \item Store cross sections and group constant data on \toapi{containers}{serpentTools.objects.containers}{HomogUniv} objects
        \end{itemize}
    \item Separate infinite-medium and leakage corrected values
    \end{itemize}
\end{frame}

\backupend

}
\end{document}
